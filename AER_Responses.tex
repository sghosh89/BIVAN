

REFEREE 1:

Review of Ghosh et al., "Copulas and their potential for ecology"

The paper by Ghosh et al provides a introduction and evaluation of a modelling technique,
copula modelling, in ecology and environmental science. The authors provide some basic
background theory as to what copulas are and what information they can summarise, and through
a series of case studies address multiple questions relating to the suitability of this technique
for modelling and understanding various ecological phenomena. As a reader who was not previously
familiar with copula modelling, I think the paper is a very useful introduction and has certainly
convinced me of the usefulness of copula modelling techniques - I will be exploring this family
of statistics further.

>>>Thank you for the positive feedback! It was our goal to get readers to see the usefulness of
copulas and consider them for their own research, so we are very happy to hear we have succeeded!

I did however find myself a little lost in certain parts, and therefore most of my comments relate
to guiding an audience who may be, statistically competent rather than confident. Overall, I
think the paper would make a nice addition to "Advances in Ecological Research" series.

>>>Thank you for the positive feedback!

- When describing the properties of the different copula families, it might be helpful to have a
figure that allows for quick and easy comparisons.

>>>Thank you for this suggestion. There were 16 families, and the properties
of copulas in each family depend on one or two parameters, depending on the
family, thus requiring multiple panels per family. This volume of pictorial
information would be extremely hard or impossible to summarize effectively in a
single figure, so we instead have several figures, each of which visually
shows properties of one family. These can be compared to give the functionality that
the referee seems to be looking for. See Figs 4-6, S3 and S7-S18. See also the new
table described in our next response.

- A minor point, but I think that subdividing the descriptions of copula properties by family could
help researchers to quickly identify the maths/parameter descriptions relevant to their chosen copula
family.

>>>We were not entirely sure what the referee was getting at here, but we
thank the referee for the comment because it spurred us to create a new table,
table 2, which summarizes the properties of the different families.

- line 318 - units should be mg

>>>Actually Mg, meaning megagrams, is correct. We clarified by replacing "Mg"
with "megagrams".

- What are the data requirements for robust copula modelling? I appreciate this
is a broad question, but generally do these methods work well for small sample
sizes, or do we need large n.

>>>Thank you for asking this question, as it prompted us to we add a new paragraph
on this topic to the Discussion. The new paragraph begins "Data requirements for
copula methods..." and is the second-to-last paragraph of the main text.

- Concepts and methods sections would be easier to read if the questions were
included, so that the reader is not forced to backtrack through the manuscript.

>>>Thanks for the suggestion, we made this change.

- The manuscript is very long and quite intimidating for a lay reader such as myself. I wonder whether reducing the number of case studies might help?

>>>Thank you for this suggestion. We have considered at various times the possibility
of splitting the manuscript up
and/or removing some of its content. In the end we decided it is best to keep it all together,
for several reasons. First, the parts of the manuscript (theoretical and empirical, the three
research question, the multiple case studies) reinforce each other, providing better evidence
for the points we are making than any of these parts would on its own. Second,
we selected datasets in an attempt to span multiple disciplines of ecology.
Removing some now would create a less representative collection 
of datasets across the field, and would weaken
our claim that non-Gaussian and informative 
copula structure seems to be a common phenomenon, broadly, across ecology.
Finally, we selected datasets prior to analysis and we present all results. This was done
to be sure to fairly evaluate the hypothesis that non-Gaussian copula structures are
common in ecology. We did *not* carry out additional analyses on other datasets and
then only present the results with interesting copula structure. We would prefer not to
modify the manuscript in such a way as to invalidate this strength of our work.

- When discussing tail-associations, it might be helpful to briefly suggest what a left or right tail association would mean in ecological terms.

>>>Tail associations are introduced in the third paragraph of the Introduction, and four 
conceptual examples of the ecological implications/meaning of copula structure and especially 
of tail associations are given in paragraphs 5-8 of the Introduction. These paragraphs 
serve to provide an initial idea of what a left- or right-tail association would mean 
in ecological terms. Because there is a paragraph between paragraphs 3 and 5, we added 
a note to paragraph 3 indicating that "We give four examples below which illustrate some of the 
potential ecological meaning of tail associations."

- Building on the above point, I think the entire results section would benefit from "touching base" with ecology more often, mainly so that readers who perhaps aren't as confident following the maths/stats can still derive the key ecological points from your work.

>>>


REFEREE 2

Review of Ghosh et al.: Copulas and the Potential for Ecology
I would like to thank the authors for their manuscript entitled "Copulas and their potential for ecology".  The manuscript provides an introduction to the concept of copulas and their utility for application in studying the natures of variable dependency in ecology.  The authors demonstrate that the most commonly used models of inference, correlation and regression, only encapsulate certain features of the dependency between variables and the author introduce a more complete method of inference to an ecologist audience.
I must first state that I must give fair warning that, whilst I have heard about copulas before, I have never applied them to my own research and I certainly cannot claim to be an expert in this area.  I am therefore probably the target audience of this paper.  This means that whilst I can give constructive feedback on the author's introduction of the concepts to an ecologist readership and can follow the mathematical background that they have provided, I cannot necessarily state definitively whether the authors have reproduced a fair portrayal of the theory of copulas in this manuscript.
However, despite not being a specialist on the subject, I found the presentation of the manuscript reasonably easy to follow and the authors should be commended for making what is quite a dense subject approachable.  Each theoretical claim is backed by well-documented supplementary materials which allows for digestion of the key concepts without being side-tracked by dense mathematical derivations.

>>>Thank you for the positive feedback.

The authors detail a large number of example ecological applications and detail a strong case for the use of copula families with left- or right- tail associations.  This is a very good demonstration where the use of copulas can benefit the ecological investigator far above the traditional methods.  

>>>Thank you for the positive feedback.

I would however appreciate a discussion of how copula-based methods compare with regression models with a heteroscedastic error term.  It seems that these models would also be able to cope with left- or right- tail associations in a bivariate application.  Given the rise of hierarchical Bayesian models in ecology, before reading this paper, my (maybe somewhat naiive) approach would be to apply a standard regression model but allow for the variance in the error distribution to also be determined by the predictor variable.

>>><DAN: to add some discussion material>

In my view this paper represents an impressive contribution to statistical ecology and I am very happy to be introduced to a new and exciting avenue of analysis.  Indeed, the thing I hate most about the manuscript is that it has spurred me on to spend hundreds of dollars on new academic books on copula modelling.

>>>Thank you for the positive feedback. This was exactly our goal!

What follows is a small set of very minor points that are all highly subjective and whether or not they are addressed does not, in my view, preclude the manuscript of being published:
o    The authors use the term "distribution function" to represent "cumulative distribution function".  My feeling is that ecologists in general are more familiar with the term "cumulative distribution function" (which thereafter can be abbreviated to "CDF") as the former is more likely to be mistaken for the density function.

>>><DAN to do>

o    The manuscript could benefit from a little more detail in the introduction as to how this material goes further than that in other ecological introductions to copulas.  For example, Anderson et al. 2018 also contains an introduction to the theory of copulas so if might be worth explicitly pointing out how this paper goes further in explaining and introducing the concept to an ecologist audience.

>>><DAN to do>

o    A table giving a run-down of the notation used in the manuscript might be useful for the more mathematically-adverse reader.  A sort of notation cheat-sheet for easy notation lookup would be useful for readers so that they can concentrate on the new concepts rather than looking up notation definitions.

>>><Shyamolina is building a preliminary version of the table>

o    The authors make reference to many families of copula models (a total of 16 families) and, whilst there is a graphical illustration of the shapes of the copulas, we don't see the functional forms of these copulas.  A table either in the main text or supplementary materials giving these would be appreciated.
 
>>><Shyamolina is collecting the formulas>
