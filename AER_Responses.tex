Hi Dan,
I have sent your Adv. Ecol. Res. submission to two peers for review. One has already returned their review (below) and I have spoken to the other who assure me it will be submitted shortly and that they are overall positive about the work. Thus I'm happy to accept this paper pending you addressing the referees' comments and thought I would send the first of these over now so that you can get cracking. I'll forward the other referee's comments shortly once I receive them. I hope this is not too onerous a task, and look forward to receiving your revisions. The final deadline for publication is January 8th 2020 - so please could you submit the final revised version then. Please note minor edits can easily be taken care of in the proof stage and it is more important to get a paper to the publishers by the 8th and then edited the proofs than it is to submit a perfect version late. I hope this makes sense. You'll also shortly receive an email from the online submission system letting you know you MS has been returned to you, this open the submission portal for the revision to be submitted. Once again, thanks for your contribution - the volume is shaping up nicely!
Cheers,
Alex
REFEREE 1:
Review of Ghosh et al., "Copulas and their potential for ecology"

%DAN: started here

The paper by Ghosh et al provides a introduction and evaluation of a modelling technique, 
copula modelling, in ecology and environmental science. The authors provide some basic 
background theory as to what copulas are and what information they can summarise, and through 
a series of case studies address multiple questions relating to the suitability of this technique 
for modelling and understanding various ecological phenomena. As a reader who was not previously 
familiar with copula modelling, I think the paper is a very useful introduction and has certainly 
convinced me of the usefulness of copula modelling techniques - I will be exploring this family 
of statistics further. 

>>>Thank you for the positive feedback! It was our goal to get readers to see the usefulness of 
copulas and consider them for their own research, so we are very happy to hear this!

I did however find myself a little lost in certain parts, and therefore most of my comments relate 
to guiding an audience who may be, statistically competent rather than confident. Overall, I 
think the paper would make a nice addition to "Advances in Ecological Research" series.

>>>Thank you for the positive feedback!

- When describing the properties of the different copula families, it might be helpful to have a 
figure that allows for quick and easy comparisons.

>>>Thank you for this suggestion. There were 16 families, and the properties 
of copulas in each family depend on one or two parameters, depending on the 
family, thus requiring multiple panels per family. This volume of pictorial 
information would be extremely hard or impossible to summarize effectively in a 
single figure, so we instead have several figures, each of which visually 
shows properties of one family. These can be compared to give the functionality that 
the referee seems to be looking for. See Figs 4-6, S3 and S7-S18. See also the new
table described in our next response.

- A minor point, but I think that subdividing the descriptions of copula properties by family could 
help researchers to quickly identify the maths/parameter descriptions relevant to their chosen copula 
family.

>>>We were not entirely sure what the referee was getting at here, but we
thank the referee for the comment because it spurred us to create a new table, 
table X, which summarizes the properties of the different families.
%***DAN: Replace X with whatever the number of the new table turns out to be-->

- line 318 - units should be mg

>>>Actually Mg, meaning megagrams, is correct. We clarified by replacing "Mg" 
with "megagrams".

- What are the data requirements for robust copula modelling? I appreciate this 
is a broad question, but generally do these methods work well for small sample 
sizes, or do we need large n.

>>>Thank you for asking this question, as it prompted us to we add a new paragraph
on this topic to the Discussio. The new paragraph begins "Data requirements for 
copula methods..." and is the second-to-last paragraph of the main text.

- Concepts and methods sections would be easier to read if the questions were 
included, so that the reader is not forced to backtrack through the manuscript.

>>>Thanks for the suggestion, we have made this change.
%***DAN: I have not done the above yet, stopped here

- The manuscript is very long and quite intimidating for a lay reader such as myself. I wonder whether reducing the number of case studies might help?

>>>

- When discussing tail-associations, it might be helpful to briefly suggest what a left or right tail association would mean in ecological terms.

>>>

- Building on the above point, I think the entire results section would benefit from "touching base" with ecology more often, mainly so that readers who perhaps aren't as confident following the maths/stats can still derive the key ecological points from your work.

>>>

